\cdpchapter{Resumen}

En este proyecto se han utilizado técnicas modernas para la segmentación de células en imágenes 3D. Actualmente los mejores resultados para tratamiento de imágenes en deep learning se obtienen con las redes neuronales convolucionales (CNN), obteniendose muy buenos resultados en segmentación celular 3D con la arquitectura UNet.

En este proyecto se han estudiado varios algoritmos y métodos para hacer segmentación celular y se han implementado aquellos que han obtenido mejores resultados en los últimos años. El desarrollo se ha basa fuertemente en \cite{Wolny2020} y \cite{Falk2019}, con una estrategia de segmentación de bordes + DT Watershed como un método y con segmentación de células añadiendo espaciado entre ellas como otro método.

Se ha desarrollado un programa sencillo que puede ser usado y modificado fácilmente. Se puede entrenar nuevos modelos y hacer inferencia con modelos ya entrenados.
