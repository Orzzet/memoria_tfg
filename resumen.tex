\cdpchapter{Resumen}

El resumen del Trabajo fin de Grado consiste, como su propio nombre indica, es un resumen de la memoria en formato apropiado 
para ser indexado en las bases de datos bibliotecarias. No debe ocupar m\'as de una carilla de texto
y en ella hay que exponer en pocas palabras la finalidad y objetivos del trabajo, 
as{\'\i} como las aportaciones realizadas. En general,
no incluir\'a figuras, cuadros ni referencias bibliogr\'aficas.

El resumen es obligatorio en espa{\~n}ol para todos los TfG. Es opcional incluir un \emph{Abstract} (resumen en ingl\'es)
en una hoja separada. Debe ser traducci\'on dle correspodiente resumen en español. En los TfG presentados en ingl\'es,
la inclusi\'on del \emph{abstract} es tmabi\'en obligatoria, as{\'\i} como en los TfG correspondientes a la titulaci\'on
de Ingenier{\'\i}a de la Salud.

El resumen de este documento es el siguiente:

Este trabajo pretende ser una gu{\'\i}a para uniformar los formatos de las memorias de los Trabajos fin de Grado de las titulaciones: 
\begin{itemize}
\item Grado en Ingenier{\'\i}a Inform\'atica --- Ingenier{\'\i}a de Computadores
\item Grado en Ingenier{\'\i}a Inform\'atica --- Ingenier{\'\i}a del Software
\item Grado en Ingenier{\'\i}a Inform\'atica --- Tecnolog{\'\i}as Inform\'aticas
\end{itemize}

de la Escuela T\'ecnica Superior de Ingenier{\'\i}a Inform\'atica de la Universidad de Sevilla.

Al mismo tiempo, se pretende que el documento sea un ejemplo de la realizaci\'on de un memoria de Trabajo fin de Grado. 
Debido a ello, hemos estructurado el documento en cap{\'\i}tulos e incluido diversos {\'\i}ndices y bibliograf{\'\i}a,
aunque obviamente no hubiera sido necesario.


