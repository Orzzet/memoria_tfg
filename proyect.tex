% !TEX encoding = UTF-8 Unicode
\documentclass{pclass}
\usepackage[utf8]{inputenc} % para linux y mac 
%\usepackage[latin1]{inputenc} % para windows

%DIFERENTES TIPOS DE LETRA
%\usepackage{palatino}
\usepackage{times}
\usepackage{subcaption}
\captionsetup{compatibility=false}

\begin{document}
\tipo{Grado}   % Grado o M\'aster
\titulopro{Segmentación de células en imágenes 3D con técnicas de machine learning}
\colaborador{Luis María Escudero Cuadrado\\Departamento de Biología Celular, Universidad de Sevilla}
\tutor{María José Jiménez Rodríguez}
\departamento{Matemática Aplicada I}
%\autores{Nombre 1}{Nombre 2}  % Dos autores
\autores{(ponente): Adrián López Carrillo}{{\ }}   % Un autor
\dia{27 de Septiembre de 2020 (v.1.1)}
\titulacion{Grado en Ingeniería Informática: Tecnologías Informáticas}

\hacerportada
	\input{tocdef}
%\frontmapertter
        
    \cdpchapter{Resumen}

En el presente proyecto se ha abordado el problema de la segmentación semántica de tejido celular. Se han investigado técnicas para resolver este problema, concluyendo que las redes convolucionales y la arquitectura UNet en concreto están en la vanguardia. Partiendo de la necesidad de disponer de una alta capacidad de procesamiento para usar aplicaciones actuales que abordan este problema, se ha desarrollado un sistema de bajo coste probado con imágenes a baja resolución. Tras completar un proceso iterativo de mejora, se ha mejorado el sistema hasta conseguir buenos resultados de segmentación semántica. Estos buenos resultados dan pie a que, en el caso de contar con mayor capacidad de procesamiento, el sistema pueda ser usado en un ámbito profesional que requiera de imágenes en alta calidad o una segmentación muy precisa.
%    \input{agradecimientos}	

	\tableofcontents % Índice de contenidos
 	\listoftables % Índice de cuadros
 	\listoffigures % Índice de figuras
% 	\lstlistoflistings %Índice de códigos

 \mainmatter  
	 
     \chapter{Presentaci\'on del problema}\label{defobjetivos}

Gracias a los avances en microscopía se provee a los investigadores una gran cantidad de datos de imágenes. Antes de que estas imágenes puedan ser analizadas por investigadores se requiere un cierto tratamiento de los datos en bruto. A veces este tratamiento requiere etiquetar manualmente miles de células o dibujar sus contornos. Este es un trabajo tedioso no muy gratificante de realizar. Por ejemplo en estudios neurocientíficos se suele requerir cuantificar el número de neuronas que expresan opsinas (un tipo de proteínas) o la localización de nuevas opsinas desarolladas en las células. Sin embargo, esta información es omitida en la mayoría de estudios a causa del esfuerzo que conlleva.

Este proyecto tiene como objetivo segmentar correctamente las células mostradas en imágenes 3D mediante la aplicación de técnicas de deep learning. Posteriormente se colorearán las células de distintos colores para facilitar su visualización. 

Este proceso podrá ser reproducido por un usuario ejecutando los archivos .ipynb con Jupyter Notebook o por línea de comandos. El usuario podrá entrenar el modelo con un nuevo conjunto de imágenes o inferir la segmentación de una o más imágenes.

Se intentará que esta herramienta sea accesible al mayor nº de usuarios posibles, para ello se optimizará todo lo posible el proceso de entrenamiento e inferencia y se estudiarán varios modelos con distinto grado de complejidad. Con esto se reducirá el hardware necesario.
	 \chapter{An\'alisis de antecedentes y aportaci\'on realizada}\label{analanteced}

%En este capítulo se hará una breve twintroducción a las redes neuronales, se justificará el uso del tipo de red neuronal CNN para tratamiento de imágenes así como el de la arquitectura CNN U-Net para la segmentación de células.

\section{Red Neuronal}\label{sec:redneuronal}
\subsection{Perceptrón}\label{subsec:perceptron}

El perceptrón es el elemento computacional básico de una red neuronal.

Fue propuesto inicialmente por Frank Rosenblatt en 1958 y perfeccionado por Minsky y Papert en la década de de 1960 \cite[p.~55-56]{Rojas1996}.
Su diseño está inspirado en las neuronas como bloque elemental en el funcionamiento del cerebro. En la figura \ref{fig:generic_computing_unit} se puede ver una unidad de computación genérica con la que puede ser descrita un perceptrón.

\figura{1}{img/Rojas1996_p31_computing_unit}{Unidad de computación genérica}{fig:generic_computing_unit}{}

Siendo:
\begin{itemize}
\item $ (x_1, x_2, ...,x_n) $ el vector de entrada.
\item $ (w_1, w_2, ...,w_n) $ el vector de pesos.
\item $ g $ la función de integración, encargada de reducir el vector de entrada a un único valor.
\item $ f $ la función de activación, encargada de producir la salida de este elemento.
\end{itemize}

El perceptrón es una unidad de computación con un umbral $ \theta $ el cual, al recibir un vector de entrada de tamaño $ n $ y dado un vector de pesos de tamaño $ n $, devuelve 1 si $ \sum_{i=1}^{n} w_i x_i \geq \theta $ y 0 en otro caso.\cite[p.~60]{Rojas1996}.

Para entender de forma intuitiva esta definición podemos tomar un ejemplo en el que $ n = 2, w_1 = 0.5, w_2 = 2, \theta = 1 $. En la figura \ref{fig:percep_graf_repr} se puede ver la línea $ 0.5x_1 + 2x_2 = 1 $ y los puntos $ p_1=(1;1), p_2=(-2;1.4) $ y $ p_3=(-0.4;-1) $. Se puede ver cómo $ p_1 $ y $ p_2 $ están por encima de la línea, por lo que devuelven 1 y $ p_3 $ por debajo, por lo que devuelve 0.

\figura{0.5}{img/percep_repr_graf}{Representación de todos los valores del vector de entrada}{fig:percep_graf_repr}{}

Se puede intuir cómo al modificar el valor de $ \theta $ la línea se mueve de forma vertical, así cómo $ x_1 $ afecta a la pendiente y $ x_2 $ afecta a ambos aspectos. Como se verá más adelante entrenar una red neuronal consiste en ir actualizando los pesos de los perceptrones, por lo que será importante representar un perceptrón de forma que se pueda modificar $ \theta $ al modificar un peso.\\

Para representar el perceptrón serán necesarios los siguientes cambios respecto a la unidad de computación genérica:
\begin{itemize}
\item Al vector de entradas se le añade un elemento de valor 1, siendo ahora de tamaño $ n+1 $.
\item Al vector de pesos se le añade un elemento de valor inicial $ -\theta $, siendo ahora de tamaño $ n+1 $. A este valor se le llamará \textit{bias}.
\item Se usará como función de integración el sumatorio.
\item Se usara como función de activación la función escalón unitario.
\end{itemize}

\figura{1}{img/Perceptron}{Perceptrón}{fig:perceptron}{}

En la figura \ref{fig:perceptron} se puede ver una representación de un perceptrón. Un perceptrón es un tipo de neurona artifical, usadas en redes neuronales artificales. Se distingirán los distintos tipos de neuronas artificiales en base a su función de activación. En el caso del perceptrón, su función de activación es la función escalón unitario, lo que lo hace especialmente útil para clasificadores binarios. Más adelante veremos el uso de otras funciones de activación y cómo es esto beneficioso para el tratamiento de imágenes.

%%

\subsection{Partes de una Red Neuronal}\label{subsec:nn_partes}
\subsection{Algoritmos de aprendizaje}\label{subsec:learning_algos}
\section{Red Neuronal Convolucional}\label{sec:cnn}
\section{Reto ImageNet}\label{sec:imagenet}
\section{Arquitectura U-Net}\label{sec:unet}     
     \input{Capitulos/datos}
     \chapter{Diseño}\label{requisitos}

En este capítulo se describirá el flujo por el que pasan los datos suministrados hasta dar lugar a la segmentación objetivo, sin entrar en detalles sobre la las herramientas usadas en la implementación.

\section{Esquema general}\label{sec:diseno-general}

El principal cuello de botella al usar técnicas de deep learning es el hardware requerido para ello. 

Sería ideal ser capaz de entrenar una CNN usando las imágenes provistas directamente, pero a causa de su gran resolución no es viable. Una imagen de resolución $ (200, 1024, 1024)vx $ con una precisión de 8 bytes ocupa en memoria (siempre hablaremos de memoria de GPU) $size(imagen)=\frac{1024*1024*200*8}{1024*1024}=1600MB $. Si bien podríamos almacenar esta imagen en memoria, las CNN se caracterizan por aplicar un gran número de filtros distintos en paralelo a una imagen de entrada, utilizando los resultados de la aplicación de estos filtros en el siguiente paso de la CNN. Tal y cómo se verá en más detalle en la sección sobre las arquitecturas seleccionadas, es completamente inviable para nosotros usar la resolución original. Es por ello que como parte del preprocesado se han simplificado los datos de entrada.

Por otro lado, también sería ideal contar con la última tecnología en GPU o TPU. En un artículo sobre segmentación de tejido celular 3D usando U-Net se explica cómo han usado 8 NVIDIA GeForce RTX 2080 Ti GPUs para realizar 100K iteraciones \ref{Wolny2020}. El acceso a un hardware superior permite el uso de arquitecturas más complejas, datos más precisos o mayor nº de iteraciones, lo que va a influir en el resultado obtenido. 

Para tener acceso a mejor hardware se usarán tecnologías cloud, donde puedes alquilar el uso de GPUs por hora siendo común pruebas gratuitas especialmente para estudiantes.

Para la parte del entrenamiento y la inferencia se usará Jupyter Notebook, esto hará que sea fácil trasladar el código entre distintas máquinas, locales o en internet.

\pagebreak \figura{1}{img/Diseno-General}{Diseño general}{fig:diseno-general}{}

En la figura \ref{fig:diseno-general} se muestra el diseño general sobre el proceso en el que los datos pasan de los archivos hdf5 inicialmente provistos hasta completar la segmentación requerida.

La etapa de preprocesado no requerirá un uso intensivo de GPU, por lo que podrá realizarse en cualquier máquina. En este caso se usará la máquina local ya que la máquina en la nube es más costosa de utilizar.

En esta etapa se preparán los datos con 4 objetivos:
\begin{enumerate}
\item Reducir la resolución de la imagen de entrada, reduciendo así la memoria necesaria para almacenarla en GPU.
\item Generar las imágenes etiquetadas correctamente para tenerlas como objetivo.
\item Reducir el tamaño de los archivos resultantes, ya que estos serán usados en servicios cloud y tendrán que ser subidos y descargados con frecuencia.
\item Modificar el orden de las dimensiones y añadir una \textit{singleton dimension} para el canal. Este formato es necesario para su uso en la CNN.
\end{enumerate}

Tras esto, se generarán nuevos archivos hdf5 y se subirán a un disco duro virtual, al cual se accederá por un Notebook. El dataset será leído por un DataLoader, el cual realizará todo el preprocesado que faltase a las imágenes de entrada, como puede ser la normalización. Las imágenes de entrada podrán entonces ser usadas como entrada en el modelo  seleccionado, cuya salida, dependiendo del modelo, podrá requerir un postprocesado o no. El resultado final será un etiquetado multiclase, que visualmente se traducirá a un coloreado de células.

\pagebreak\section{Preprocesado local}\label{sec:preprocesado-local}

\figura{1}{img/Diseno-Preprocesado}{Preprocesado llevado a cabo en la máquina local. Los valores redimensionados son }{fig:preprocesado}{}

En la figura \ref{fig:preprocesado} se puede ver con más detalle el resultado que se obtendría en esta etapa. 

\subsubsection{Reducir la resolución de la imagen de entrada}

El formato de las imágenes inicialmente es $ (Z,X,Y) $, siendo para todas las imágenes $ X = Y = 1024 $ y $ Z\epsilon[216,368] $. Además de reducir la resolución de las imágenes, es importante que todas tengan la misma, de lo contrario no podrán ser usadas en operaciones batch. También es esencial que la imagen de entrada y la imagen etiquetada no se deformen demasiado.

\subsubsection{Generar las imágenes etiquetadas correctamente}

El problema principal que nos encontramos en las imágenes con el etiquetado perfecto es que las células, siendo instancias de una misma clase (la clase "célula"), tienen etiquetas distintas. En una segmentación semántica cada vóxel es etiquetado con la clase a la que se cree que pertenece. Si usáramos el etiquetado actual necesitaríamos una clase por cada célula, pero eso no tiene mucho sentido ya que el nº de células en una imagen puede variar, además todas las células tienen características similares. Probaremos dos soluciones a este problema:

\begin{enumerate}
\item Añadir espacio entre células para que ninguna esté en contacto y hacer que todas tengan $ 1 $ como etiqueta.
\item Cambiar las etiquetas a $ 1 $ sin añadir espaciado y entrenar un modelo A para su predicción. Encontrar los bordes exteriores de las células y entrenar un modelo B para su predicción. La predicción final será la predicción del modelo A menos la predicción del modelo B.
\end{enumerate}

Los 3 etiquetados distintos son preprocesados en este paso.

\subsubsection{Reducir el tamaño de los archivos resultantes}

Al estar trabajando con herramientas en la nube, será recomendable reducir el tamaño de los archivos lo mayor posible.

Todas las imágenes tienen una precisión de 8 bytes, cuando en un estudio previo se concluyó que no era necesaria tanta precisión para los valores de esas imágenes, especialmente para el etiquetado que usa valores enteros entre $0$ y $100$. Es importante tener en cuenta que al almacenar tensores en la memoria de la GPU, la imagen no tendrá ningún tipo de compresión, cada elemento ocupará espacio en memoria. Si hacemos que la imagen de entrada pase de 8 bytes a 4 bytes, estaremos reduciendo su tamaño a la mitad. De forma similar si hacemos que la imagen del etiquetado pase de 8 bytes a 1 bytes, estaremos reduciendo su tamaño a una octava parte. Con 1 byte de precisión podremos almacenar hasta 256 valores distintos, suficiente ya que sólo tendremos 2 valores distintos: 0 para el fondo y 1 para la célula o los bordes.

Además se comprimirán las imágenes, aunque esto sólo reducirá el tamaño del archivo, el tamaño de los tensores almacenados en memoria será el mismo. Esta compresión será muy efectiva en las imágenes etiquetadas, ya que los elementos sólo tendrán 2 valores distintos.

\subsubsection{Modificar el orden de las dimensiones}

El cambio de las dimensiones se debe principalmente a las herramientas usadas en etapas posteriores, que requieren los ejes ordenados como $(x,y,z)$.

Añadir una \textit{singleton dimension} es necesario para tener en cuenta el canal de la imagen, ya que esto es usado en los componentes de la CNN con arquitectura U-Net.
     \chapter{Implementación}\label{implementacion}

\section{Carga de datos}\label{sec:data_loading_processing}
\begin{itemize}
\item \textbf{TorchIO.} 

\end{itemize}
Además, después de la ZNormalization, se han normalizado los valore al rango $[0.0, 1.0]$.  

\subsection{Consideraciones}

\subsubsection{Tamaño del batch}

Se han hecho pruebas con varios tamaños de batch. El tamaño del batch determina cuántos ejemplos van a usarse en el entrenamiento del modelo. La entrada del modelo será de dimensión $(B,1,X,Y,Z)$, donde $B$ es el tamaño del batch. La ventaja de tener una entrada con este formato es se realizarán el mismo nº de operaciones matriciales sin importar el tamaño del batch. 
En el modelo MiniUnet3D al usar un batch de 1 los epochs tardan 11 segundos y al usar un batch de 4 tardan 7 segundos. Por cada batch utilizado se necesitan $\sim 3GB$ de memoria, por lo que se tomó como una buena opción para mejorar la velocidad de entrenamiento, aunque los resultados de este modelo eran malos ya que es una arquitectura reducida Unet3D.
Al entrenar con el modelo Unet3D, se comprobó que el nº de batch máximo posible era 2 debido a que se requiere más memoria al haber más capas. Con un batch de 1 tarda 39 segundos por epoch, con un batch de 2 tarde 34 segundos. Sin embargo, los resultados para un batch de 1 eran mejores, por lo que al se optó por usar un batch de 1, provocando que el tamaño de las imágenes usadas en el entrenamiento no tenga por qué ser el mismo.


\section{U-Net}\label{sec:unet_implementation}

\subsubsection{Funciones de pérdida}
Se han probado 3 funciones de pérdida: la entropía cruzada, la entropía cruzada con pesos y la función de pérdida DICE.

La entropía cruzada se descartó rápidamente ya que, al ser las dos clases muy desequilibradas daba demasiada prioridad al fondo sobre la segmentación y se tendía a perder mucha segmentación. 

Tras esto se probó la pérdida DICE. Para ello se utilizó la función \textit{Softmax} en la salida del modelo (logits) para que la suma de las probabildades de que un vóxel pertenezca al fondo y a la segmentación sume 1. El resultado y el etiquetado perfecto se pasaron al formato \textit{one hot} y se compararon con la función de pérdida DICE. Esto dio muy buenos resultados, consiguiendo por primera vez un $IoU>0.7$ para la segmentación.

Por último se probó la entropía cruzada con pesos, dando como peso a la clase de segmentación $1$ y se probó darle a la clase borde $0$, $0.1$ y $0.2$, sin obtener buenos resultados en ningún caso. En el artículo estudiado en los antecedentes \cite{Falk2019} se usó entropía cruzada con pesos dando dando pesos distintos a cada vóxel. No se llegó a probar esta implementación, que podría haber sido buena para la segmentación con espaciado entre células.

\subsubsection{Precisión Mixta}



%     \input{Capitulos/alternativas}
     \chapter{Pruebas}\label{pruebas}

Pongo sólo las imágenes por ahora. Todas las métricas están guardadas.

\figura{1}{img/unet4.12-200e-LQNOB-norm-Znorm}{Arquitectura U-Net media. Batch=1. Znorm. Hay sobreajuste.}{fig:unetarch-half}{}
\figura{1}{img/unet4.2-200e-LQNOB-norm-Znorm-AdamConParametrosCorrectos}{Arquitectura U-Net media. Batch=1. Adam con mejores parámetros. Hay sobreajuste.}{fig:unetarch-half}{}
\figura{1}{img/unet4.8-200e-aug}{Arquitectura U-Net media. Con data augmentation. Data augmentation elimia el sobreajuste.}{fig:unetarch-half}{}
\figura{1}{img/unet5.2-losses-400e-boundaries-apex}{Arquitectura U-Net completa. Bordes. Data augmentation. Apex. El autoescalado del factor de pérdida de Apex aumenta la velocidad de entrenamiento.}{fig:unetarch-half}{}
\figura{1}{img/unet5.2-metrics-400e-boundaries-apex}{Arquitectura U-Net completa. Bordes. Data augmentation. Apex.}{fig:unetarch-half}{}
\figura{1}{img/unet5.2-sample-400e-boundaries-apex}{Arquitectura U-Net completa. Bordes. Data augmentation. Apex.}{fig:unetarch-half}{}
\figura{1}{img/unet5.3-losses-200e-target-apex}{U-Net completa. Espaciado. Data augmentation. Apex}{fig:unetarch-half}{}

\figura{1}{img/unet5.3-sample-200e-target-apex}{U-Net completa. Espaciado. Data augmentation. Apex}{fig:unetarch-half}{}
	
\figura{1}{img/unet5.4-losses-200e-not_segmented-apex}{U-Net completa. Sin espaciado. Data augmentation. Apex}{fig:unetarch-half}{}
	
\figura{1}{img/unet5.4-metrics-200e-not_segmented-apex}{U-Net completa. Sin espaciado. Data augmentation. Apex}{fig:unetarch-half}{}
	
\figura{1}{img/unet5.4-sample-200e-not_segmented-apex}{U-Net completa. Sin espaciado. Data augmentation. Apex}{fig:unetarch-half}{}
	



     \chapter{An\'alisis temporal y de costes}\label{anatemporal}

\section{Análisis temporal} 
\cuadro{|c|c|}{Cuadro resumen de la planificación.}{tab:prueba}{
Fecha de inicio & 20/02/2020 \\ 
\hline
Fecha de fin prevista & 14/06/2020 \\
\hline
Fecha de fin real & 4/12/2020 \\
\hline
Horas de trabajo previstas & 300\\
\hline
Horas de trabajo real & \\
}

\cuadro{|c|c|c|}{Cuadro resumen de los costes.}{tab:costes}{
Concepto & Coste previsto & Coste real \\ 
\hline
Coste de personal & $5074.22$ & \\
\hline
Coste de servidores & $65.81$ & \\
\hline
Costes indirectos & $660.36$ & $1650.9$\\
\hline
Total & $5800.39$ &  \\
}


\subsection{Costes directos}\label{sec:costesdirectos}

\subsubsection{Costes de personal}\label{subsec:personal}

En los costes de personal se ha incluido el salario de los trabajadores involucrados en el proyecto. 

Tras investigar salarios actuales se ha concluido que un junior en el ámbito de inteligencia artificial (Ingeniero de Machine Learning, Ingeniero de Inteligencia Artificial, Ingeniero de Datos, Ingeniero de Big Data) puede cobrar $25000 $€$ $ brutos anuales. Según el BOE \cite{BOE} la empresa deberá pagar un $29.90\%$ ($23.60\%$ contigencias comunes + $5.50\%$ desempleo + $0.20\%$ FOGASA + $0.60\%$ formación profesional) del salario como coste a la Seguridad Social. Haciendo un total de $25000 $€$ * 1.299 = 32475 $€ anual.

Suponiendo que el sueldo se cobra en 12 pagas y que la jornada laboral es de 160 horas, el coste por hora para la empresa sería: $32475 $€$ / (12 * 160) = 16.914$€.

Al ser un proyecto de 300 horas, los costes de personal en total son $16.914 $€$ * 300 = 5074.22 $ €

\subsubsection{Costes de servidores}\label{subsec:servidores}

Al tratarse este proyecto de una aplicación de visión artificial con imágenes de alta definición y deep learning, el coste computacional ha sido muy alto. Es por ello que para el desarrollo de este sistema ha sido necesario el uso de una GPU con al menos 16GB de memoria (VRAM) para poder llegar a unos resultados mínimos. Al no disponer de ninguna máquina en físico que cumpla este requisito, se ha optado por alquilar servidores virtuales.

La computación se ha realizado en una máquina con la GPU P5000, con un valor de mercado de $1890.63$€ \cite{amazonp5000} pudiéndose alquilar por $0.78$\$/hora \cite{gradientdocs}. Aunque tenga estos costes, se ha aprovechado que la web en la que se puede alquilar también ofrece una capa gratuita en la que se puede usar esa GPU de forma indefinida con algunas limitaciones. Aún así se ha hecho el cálculo sin tener esta capa gratuita.

El servidor se ha usado aproximadamente durante 100 horas, usándose para desarrollar, entrenar y hacer inferencia. El coste sería de $0.78 * 100 = 78$\$, que en el momento de escribir esto equivale a $65.81$€.

\subsection{Costes indirectos}\label{sec:costesindirectos}

Se tomará como costes indirectos el uso de material informático en físico y costes varios.

Se ha utilizado el ordenador portátil MSI GP62 7RE Leopard Pro con un valor actual de $905.59$€ \cite{pccomponentes}. Suponiendo que un portátil tiene una esperanza de vida de 5 años, cada mes que se use en el proyecto supondrá un coste de $905.59 / (12*5) = 15.09$€.

Como costes varios se tomará la electricidad, el uso de periféricos, el coste de la oficina y otros. Debido a la incertidumbre al hacer este cálculo se supondrá un gasto de $150$€ al mes.

Según el plan en el que se estimaban 4 meses, el coste sería de $4 * (15.09 + 150) = 660.36$€. En la realidad el proyecto ha tomado 10 meses, por lo que el coste real sería $10 * (15.09 + 150) = 1650.9$€.

\cuadro{|c|c|c|}{Tabla de tiempos}{tab:tiempos}{
Tarea & Subtarea & Tiempo \\
\hline
Configurar servicios Cloud  &										& 16h64m \\
\hline
Investigar sobre CNN	    &										& 59h3m \\
\hline
Primer sistema				& Arquitectura MiniUnet3D				& 1h30m \\
							& Función de Pérdida y Optimizador		& 4h10m \\
							& Simplificar CNN						& 9h40m \\
							& Resultados primer entrenamiento		& 4h22m \\
							& Espaciado entre células				& 12h18m \\
							& Leer dataset en PyTorch				& 3h15m \\
							& Estudiar imágenes de entrada			& 5h22m \\
							& Otros									& 12h33m \\
							& Añadir métricas de validación y test	& 2h59m \\
							& Métrica IoU							& 2h55m \\
							& Preparar datos para batch				& 2h37m \\
\hline
Pruebas Plantseg			& 										& 5h45m \\
\hline
Mejoras					& Otras									& 4h22m \\
							& Pérdida DICE							& 3h6m \\
							& Data augmentation						& 3h19m \\
							& Normalización							& 6h40m \\
							& UNet3D completa						& 2h58m \\
							& Implementar Apex (precisión mixta)	& 4h49m \\
							& Watershed								& 5h31m \\
							& Pruebas sin éxito						& 3h1m \\
\hline
Conversor a tif para fiji	& 										& 5h20m \\
\hline
Representación visual		& 										& 6h9m \\
\hline
Memoria						& 										& 108h12m \\
\hline
Preparar Proyecto			& 										& 7h39m \\
\hline
Total						&										& 304h46m\\
}
     \chapter{Conclusiones}\label{pruebas}

Durante el desarrollo de este trabajo han sido dos las principales dificultades que se han encontrado: \textbf{conocimiento en la materia} y \textbf{capacidad de computación}.

Podría parecer intuitivo pensar que la inteligencia artificial es la que soluciona un problema al aplicar un algoritmo. Sin embargo, en este trabajo se ha experimentado que esto no es así. Es necesario un amplio conocimiento en varios dominios para afrontar un problema real. Este proyecto se inició con la idea de solucionar el problema de segmentación celular utilizando todas las herramientas disponibles para ello y rápidamente se descubrió que los algoritmos de deep learning están a la vanguardia, en concreto la arquitectura U-Net. Se intentó recurrir a soluciones de segmentación celular ya existentes sin éxito, ya que estas soluciones no estaban especializadas en resolver exactamente el mismo problema o requerían de un hardware muy superior al disponible. Quedando como única opción una implementación más a bajo nivel.

Los resultados obtenidos al realizar 100 o 300 iteraciones entrenando el modelo UNet con imágenes de baja calidad son consideradas buenos ($IoU>0.7$), consiguiendo con el mejor modelo $IoU\sim0.75$. Además, el programa desarrollado se puede usar sin hacer ningún cambio con imágenes de mayor calidad o iterando un mayor número de veces, lo que creemos mejoraría mucho los resultados. En los artículos seguidos se han hecho 100k y 150k iteraciones sobre imágenes de mayor calidad, obteniendo a veces unos resultados casi perfectos. Con 100k iteraciones en el algoritmo desarrollado probablemente se consigan resultados comparables con los obtenidos en esos artículos.

     \chapter{Manual}\label{manual}

El proyecto está disponible en el siguiente enlace:

\url{https://github.com/Orzzet/segmentacion_celular_unet3d}. 

\section{Instalación}

Prerequisitos:

\begin{itemize}
\item Sistema operativo Linux.
\item CUDA \url{https://docs.nvidia.com/cuda/cuda-installation-guide-linux/index.html}
\item Python 3.5+ \url{https://www.python.org/downloads/}
\item (Opcional) Jupyter Notebook \url{https://jupyter.readthedocs.io/en/latest/install.html} o un software que pueda ejectur Notebooks.
\end{itemize}


Instalación:

\begin{verbatim}

pip install torch torchvision
git clone https://github.com/Orzzet/3d-cell-segmentation.git
cd 3d-cell-segmentation
pip install requirements.txt

\end{verbatim}

\section{Guía de uso}
El proyecto puede utilizarse por consola de comandos o por Notebooks.

\subsubsection{Consola de comandos}

Para facilitar los parámetros de configuración se usa un archivo .yaml (como el archivo \url{https://github.com/Orzzet/3d-cell-segmentation/blob/master/config.yaml}) con la siguiente estructura:
\begin{verbatim}
# train, test o predict
job: "test" 
# Cociente por el que se divide cada dimensión.
dim_size_reduction: [1,1,1]
# Nombre del dataset etiquetado dentro del archivo .h5
target_mode: "target" 
# True usa gpu, False usa cpu
gpu: True
# Nombre del modelo (sin extensión ni path)
model_name: "prueba"
# Path a la carpeta donde se guardan los modelos 
models_folder: "./"
# Arquitectura a usar. UNet3D o MiniUNet3D. 
arch: "UNet3D"
# True usa precisión mixta, False no. 
AMP: True

# Si job=train:
train_path: "data/cells/0.25z 0.125x 0.125y/train/"
valid_path: "data/cells/0.25z 0.125x 0.125y/valid/"
loss_function: "dice"
n_epochs: 100

# Si job=test:
test_path: "data/cells/0.25z 0.125x 0.125y/test/"

# Si job=predict:
input_path: "data/cells/0.25z 0.125x 0.125y/test/"
output_path: "data/cells/0.25z 0.125x 0.125y/test/prediction"
\end{verbatim}

Después de configurar este archivo, abrir un terminal, navegar hacia la raíz del proyecto y escribir:

\begin{verbatim}
python main.py "PATH/TO/YAML"
\end{verbatim}

Donde PATH/TO/YAML es el path hacia del archivo de configuración .yaml

\subsubsection{Jupyter Notebook}

Será necesario haber instalado el prerequisito opcional Jupyter Notebook. Tras esto ejecutar Jupyter Notebook y navegar hasta la carpeta raíz del proyect.

Para entrenar un nuevo modelo usar el notebook 1\_Entrenamiento.ipynb. Hay que configurar los siguientes parámetros, siendo estos parámetros los mismos que los usados por consola de comandos:

\begin{verbatim}
DIM_SIZE_REDUCTION = (1,1,1)
MODELS_FOLDER = "../models/"
TARGET_MODE = 'target'
model_name: "prueba"
n_epochs = 300
train_path = '../data/cells/0.25z 0.125x 0.125y/train/'
valid_path = '../data/cells/0.25z 0.125x 0.125y/valid/'
arch = "MiniUNet3D"
loss_function = "dice"
AMP = False
\end{verbatim}

Tras configurar estos parámetros, ejecutar todas las celdas. Si el nombre del modelo ya existe en el path de modelos indicado, se reanudará el entrenamiento por el último epoch entrenado.

Para calcular métricas de un modelo, usar el notebook 2\_Test.ipynb. Para usar este notebook hay que configurar los mismos parámetros que en el entrenamiento, a excepción que en este notebook hay que indicar el path hacia el conjunto de test:\\

\verb|test_path = '../data/cells/0.25z 0.125x 0.125y/test/'|\\

Para predecir una segmentación, usar el notebook 3\_Infe.ipynb. Hay que indicar el directorio donde se encuentran las imágenes originales y el directorio donde guardar los etiquetados obtenidos. La predicción se guardará en un archivo con formato .h5, dentro de este archivo se guardará en ``prediction''. Además se mostrarán 4 capas de la segmentación obtenida de cada imagen de entrada.

\backmatter

% \input{Capitulos/apendices}

\bibliographystyle{apacite}

\bibliography{pfcbib}

\end{document}