\chapter{Objetivos}

Este proyecto comenzó con la idea de reducir el tiempo necesario para tener una segmentación correcta de las células, por lo que ese será el objetivo principal. En las etapas iniciales del proyecto, tal y como se verá en el capítulo \ref{planseguido} hubo un cambio de rumbo en el alcance del proyecto, es por ello que se añadió un objetivo relacionado con el coste de computación. En definitiva, los objetivos son los siguientes:

\begin{enumerate}
\item Utilizar técnicas de machine learning para desarrollar un modelo que, a partir de una imagen 3D de tejido epitelial, calcule una segmentación de las células lo más cercano posible a una segmentación perfecta.
\item Debido a que especies con distintos fenotipos pueden presentar glándulas con distinta morfología a nivel celular, es útil la posibilidad de poder entrenar un modelo con distintos datasets.
\item Debido a las dificultades encontradas en el primer acercamiento al problema, hacer que el proyecto funcione correctamente con una capacidad de computación limitada.
\item Debido a que el proyecto será utilizado por personal del ámbito científico y no tecnológico, hacer que el proyecto sea de uso e instalación sencilla.
\end{enumerate}