\chapter{Conclusiones}\label{pruebas}

Durante el desarrollo de este trabajo han sido dos las principales dificultades que me he encontrado: \textbf{conocimiento en la materia} y \textbf{capacidad de computación}.

Podría parecer intuitivo pensar que la inteligencia artificial es la que soluciona un problema al aplicar un algoritmo. Sin embargo he experimentado que esto no es así. Es necesario un amplio conocimiento en varios dominios para afrontar un problema real. Empecé el proyecto intentando solucionar el problema de segmentación celular utilizando todas las herramientas disponibles para ello y rápidamente descubrí que los algoritmos de deep learning están a la vanguardia, en concreto la arquitectura U-Net. Intenté recurrir a soluciones de segmentación celular sin éxito, ya que estas soluciones no estaban especializadas en resolver exactamente el mismo problema o requerían de un hardware muy superior al disponible. Quedando como única opción una implementación más a bajo nivel.

Los resultados obtenidos al realizar 100 o 200 iteraciones entrenando el modelo UNet con imágenes de baja calidad son consideradas buenos ($IoU>0.7$). Se puede usar el programa desarrollado sin hacer ningún cambio con imágenes de mayor calidad o iterando un mayor nº de veces, lo que mejorará mucho los resultados. En los artículos seguidos se han hecho 100k y 150k iteraciones sobre imágenes de mayor calidad, obteniendo a veces unos resultados casi perfectos. Con 100k iteraciones en el algoritmo desarrollado probablemente se consigan resultados comparables con los obtenidos en esos artículos.
