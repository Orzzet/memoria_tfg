\chapter{Conclusiones}\label{pruebas}

Durante el desarrollo de este trabajo han sido dos los principales retos que se han superado: \textbf{conocimiento en la materia} y \textbf{capacidad de computación}.

Podría parecer intuitivo pensar que con la inteligencia artificial se puede resolver un problema aplicando un algoritmo sin más. Sin embargo, en este trabajo se ha experimentado que esto no es así. Es necesario un amplio conocimiento en varios dominios para afrontar un problema real. Este proyecto se inició con la idea de solucionar el problema de segmentación celular utilizando todas las herramientas disponibles para ello y rápidamente se descubrió que los algoritmos de deep learning están a la vanguardia, en concreto la arquitectura U-Net. Se intentó recurrir a soluciones de segmentación celular ya existentes sin éxito, ya que estas soluciones no estaban especializadas en resolver exactamente el mismo problema o requerían de un hardware muy superior al disponible. Esto llevó al planteamiento de buscar una solución propia más a bajo nivel, aunque inspirada en las existentes, que es justamente lo que se ha conseguido.

Los resultados obtenidos al realizar 100 o 300 iteraciones entrenando el modelo UNet con imágenes de baja calidad son buenos ($IoU>0.7$), consiguiendo con el mejor modelo $IoU\sim0.75$. Con un etiquetado hecho por una persona se suele conseguir $IoU\sim0.8$, por lo que al conseguir $0.75$ con el sistema desarrollado en tan solo 300 iteraciones es probable que con un entrenamiento más extenso se consigan resultados superiores a los que conseguiría una persona sin la desventaja del tiempo invertido en hacer ese etiquetado. Además, el sistema desarrollado se puede entrena con imágenes de alta calidad sin hacer ningún cambio en el código. En los artículos estudiados se han hecho 100k y 150k iteraciones sobre imágenes de mayor calidad, obteniendo a veces unos resultados casi perfectos. Entrenando el sistema desarrollado con imágenes de alta resolución y 100k iteraciones, probablemente se consigan resultados comparables con los obtenidos en esos artículos, consiguiendo así que el sistema desarrollado sea útil en un ámbito profesional en el que se necesiten imágenes de alta resolución o una segmentación muy precisa.

