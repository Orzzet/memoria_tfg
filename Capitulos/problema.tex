\chapter{Presentaci\'on del problema}\label{defobjetivos}

Gracias a los avances en microscopía se provee a los investigadores una gran cantidad de datos de imágenes. Antes de que estas imágenes puedan ser analizadas por investigadores se requiere un cierto tratamiento de los datos en bruto. A veces este tratamiento requiere etiquetar manualmente miles de células o dibujar sus contornos. Este es un trabajo tedioso no muy gratificante de realizar. Por ejemplo en estudios neurocientíficos se suele requerir cuantificar el número de neuronas que expresan opsinas (un tipo de proteínas) o la localización de nuevas opsinas desarolladas en las células. Sin embargo, esta información es omitida en la mayoría de estudios a causa del esfuerzo que conlleva.

El Departamento de Biología Celular de la Facultad de Biología de la Universidad de Sevilla tenía un problema similar. Necesitaban un método para segmentar con precisión y velocidad imágenes tomadas por un microscopio confocal con gran detalle. Tras probar varias técnicas decidieron intentar hacerlo utilizando herramientas de machine learning. El procesado digital de estas imágenes actualmente se hace de forma manual teniendo una duración de una a dos semanas, por lo que la automatización de este proceso conllevará un gran ahorro de tiempo.