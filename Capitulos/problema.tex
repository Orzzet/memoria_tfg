\chapter{Presentaci\'on del problema}\label{defobjetivos}

Gracias a los avances en microscopía se provee a los investigadores una gran cantidad de datos de imágenes. Antes de que estas imágenes puedan ser analizadas por investigadores se requiere un cierto tratamiento de los datos en bruto. A veces este tratamiento requiere etiquetar manualmente miles de células o dibujar sus contornos. Este es un trabajo tedioso no muy gratificante de realizar. Por ejemplo en estudios neurocientíficos se suele requerir cuantificar el número de neuronas que expresan opsinas (un tipo de proteínas) o la localización de nuevas opsinas desarolladas en las células. Sin embargo, esta información es omitida en la mayoría de estudios a causa del esfuerzo que conlleva.

Este proyecto tiene como objetivo segmentar correctamente las células mostradas en imágenes 3D mediante la aplicación de técnicas de deep learning. Posteriormente se colorearán las células de distintos colores para facilitar su visualización. 

Este proceso podrá ser reproducido por un usuario ejecutando los archivos .ipynb con Jupyter Notebook o por línea de comandos. El usuario podrá entrenar el modelo con un nuevo conjunto de imágenes o inferir la segmentación de una o más imágenes.

Se intentará que esta herramienta sea accesible al mayor nº de usuarios posibles, para ello se optimizará todo lo posible el proceso de entrenamiento e inferencia y se estudiarán varios modelos con distinto grado de complejidad. Con esto se reducirá el hardware necesario.