\chapter{Manual}\label{manual}

El proyecto está disponible en el siguiente enlace, donde hay instrucciones sobre su instalación \url{https://github.com/Orzzet/segmentacion_celular_unet3d}.

Prerequisitos:
\begin{itemize}
\item CUDA \url{https://docs.nvidia.com/cuda/cuda-installation-guide-linux/index.html}
\item CUDNN \url{https://docs.nvidia.com/deeplearning/cudnn/install-guide/index.html}
\item Python 3.5+ \url{https://www.python.org/downloads/}
\item PyTorch \url{https://github.com/pytorch/pytorch}
\item Jupyter Notebook \url{https://jupyter.readthedocs.io/en/latest/install.html} o un Software que pueda ejectura Notebooks
\end{itemize}

Cómo usarlo:
\begin{enumerate}
\item Descargar el proyecto del repositorio y descomprimir en el lugar deseado.
\item El notebook 0\_preprocesado trata las imágenes y las transforma en el formato correcto. En caso de tener imágenes en un formato distinto, hay que modificar este fichero.
\item El notebook 1\_procesado es para hacer el entrenamiento.
\item El notebook 2\_inferencia es para utilizar el modelo, bien para estadísticas o para comprobar el resultado.
\end{enumerate}


