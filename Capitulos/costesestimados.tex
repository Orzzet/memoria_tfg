\chapter{Costes}\label{costes}

\section{Resumen}\label{sec:resumencostes}
\cuadro{|c|c|c|}{Tabla de resumen de los costes.}{tab:costes}{
Concepto & Coste previsto & Coste real \\ 
\hline
Coste de personal & $5074.22$ & \\
\hline
Coste de servidores & $65.81$ & \\
\hline
Costes indirectos & $660.36$ & $1650.9$\\
\hline
Total & $5800.39$ &  \\
}


\section{Costes directos}\label{sec:costesdirectos}

\subsection{Costes de personal}\label{subsec:personal}

En los costes de personal se incluirá el salario de los trabajadores involucrados en el proyecto. 

Tras investigar salarios actuales he concluido que un junior en el ámbito de inteligencia artificial (Ingeniero de Machine Learning, Ingeniero de Inteligencia Artificial, Ingeniero de Datos, Ingeniero de Big Data) puede cobrar $25000 $€$ $ brutos anuales. Según el BOE \cite{BOE} la empresa deberá pagar un $29.90\%$ ($23.60\%$ contigencias comunes + $5.50\%$ desempleo + $0.20\%$ FOGASA + $0.60\%$ formación profesional) del salario como coste a la Seguridad Social. Haciendo un total de $25000 $€$ * 1.299 = 32475 $€ anual.

Suponiendo que el sueldo es cobra en 12 pagas y la que la jornada laboral es de 160 horas, el coste por hora para la empresa sería: $32475 $€$ / (12 * 160) = 16.914$€ .

Al ser un proyecto de 300 horas, los costes de personal en total son $16.914 $€$ * 300 = 5074.22 $ €

\subsection{Costes de servidores}\label{subsec:servidores}

Al tratarse este proyecto de una aplicación de visión artificial con imágenes de alta definición y deep learning, el coste computacional es muy alto. Es por ello que para el desarrollo de este sistema es necesario el uso de una GPU con con al menos 16GB de memoria (VRAM) para poder llegar a unos resultados mínimos. Al no disponer de ninguna máquina en físico que cumpla este requisito (especialmente ya que este proyecto ha sido desarrollado en 2020 con las limitaciones que eso conlleva), se ha optado por alquilar servidores virtuales.

El desarrollo se hará en una máquina con la tarjeta gráfica P5000, con un valor actual de mercado de $1890.63$€ \cite{amazonp5000} que se puede alquilar por $0.78$\$/hora \cite{gradientdocs}. Aunque tenga estos costes, se aprovechará que en la web que se puede alquilar también ofrece una capa gratuita en la que se puede usar esa GPU de forma indefinida con algunas limitaciones. Aún así se hará el cálculo sin tener esta capa gratuita.

Suponiendo que se usará este servidor durante 100 horas, ya sea para desarrollar, entrenar o hacer inferencia, el coste sería de $0.78 * 100 = 78$\$, que a día de hoy son $65.81$€.

\section{Costes indirectos}\label{sec:costesindirectos}

Se tomará como costes indirectos el uso de material informático en físico y costes varios.

Se ha utilizado el ordenador portátil MSI GP62 7RE Leopard Pro con un valor actual de $905.59$€ \cite{pccomponentes}. Suponiendo que un portátil tiene una esperanza de vida de 5 años, cada mes que se use en el proyecto supondrá un coste de $905.59 / (12*5) = 15.09$€.

Como costes varios se tomará la electricidad, el uso de periféricos, el coste de la oficina y otros. Debido a la incertidumbre al hacer este cálculo se supondrá un gasto de $150$€ al mes.

Según el plan en el que se estimaban 4 meses, el coste sería de $4 * (15.09 + 150) = 660.36$€. En la realidad el proyecto ha tomado 10 meses, por lo que el coste real sería $10 * (15.09 + 150) = 1650.9$€.