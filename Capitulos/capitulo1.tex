% !TEX root = ../proyect.tex

\chapter{Formato}\label{cap1}
\section{Introducci\'on}\label{sec:intro}

La memmoria del Trabajo fin de Grado se entrega utilizando la aplicaci\'on de la escuela para 
Trabajos fin de Carrera, sita en 
\url{https://tfc.eii.us.es/TfG/}. Debe entregarse en formato \textsc{pdf} y podr\'a utilizarse
cualquier aplicaci\'on inform\'atica para generar la misma, siempre que el resultado
final verifique los requisitos
que se exponen en este documento.


\section{Formato general}\label{sec:general}

Salvo excepciones justificadas, el \textsc{pdf} 
a entregar debe tener sus p\'aginas del tama{\~n}o
estandarizado m\'as com\'un en Europa: \textsc{din a4}. Con el objetivo de facilitar 
su lectura en dispositivos electr\'onicos, se considerará que
se imprime a una s\'ola cara (ver \ref{sec:impresion} para los casos en que la memoria deba ser impresa).


\section{Cuerpo y tipos de letra}\label{sec:letra}

Se recomienda utilizar, dada su simplicidad, claridad y legibilidad, los tipos de letra
Arial (preferiblemente en caja alta) o Helv\'etica, en un tamaño para el cuerpo del texto de 11 pt,
con un interlineado sencillo o de 1,5.

Los t{\'\i}tulos de cap{\'\i}tulos, secciones y subsecciones, as{\'\i} como las notas al pie de texto\footnote{Por favor, no abusad de las notas a pie de texto}
y las cabeceras o pies de cuadros, figuras y trozos de c\'odigo quedan a libertad del redactor de la memoria.
Sin embargo, es buena idea que los t{\'\i}tulos tengan un tamaño igual o superior a 11pt. y los pies y cabeceras
sean de tama{\~n}o igual o inferior a 11pt. Se ruega encarecidamente que, en lo posible, se evite el \underline{subrayado}.
Este puede sustituirse por el uso de \textbf{negritas} o \textit{cursivas} o por el cambio de formato del tipo de letra
(Se recomienda prestar atenci\'on y no abusar del cambio de {\color{blue}{color}}, 
pues puede dar problemas de accesibilidad).

 

\section{Márgenes y párrafos}\label{sec:margenes}

Es necesario configurar la página seleccionando los márgenes siguientes:
\begin{itemize}
 \item Márgenes superior e inferior: 2,5 cm. 
 \item Márgenes laterales (izquierdo y derecho): 3 cm.
\end{itemize}

En el caso de no usar cabeceras, las páginas deber\'an ir numeradas en el centro del pie.
Si se usan cabeceras, no se utilizarán estas en las p\'aginas que comienzan capítulo, las cuales
se numerar\'an en el centro del pie. Las restantes p\'aginas pueden ir numeradas en el pie o en la cabecera,
pero deber\'an mantener coherencia de formato a lo largo de todo el cap{\'\i}tulo.

Las p\'aginas previas al cuerpo de la memoria del Trabajo fin de Grado (agradecimientos, resumen, {\'\i}ndices,\dots) 
pueden no numerarse o
numerarse independientemente de la misma, en cuyo caso se numerar\'an con n\'umeros romanos. 
Se recuerda que los n\'umeros romanos
se escriben con letras may\'usculas (la numeraci\'on i,ii,iii,iv\dots es propia del idioma ingl\'es 
y no es admisible en espa{\~n}ol, ni siquiera para enumeraciones)

Los p\'arrafos comenzar\'an con sangrado. El espacio entre los mismos no debe ser excesivo.

\section{Lengua}\label{sec:lengua}

La Normativa académica de los Trabajos fin de Grado indica:

\emph{Como norma general, el TfG deberá estar escrito y ser expuesto oralmente en castellano.
Podrá también estar escrito y ser expuesto en inglés, previa solicitud.}

En cualquier caso, la memoria debe respetar los usos y costumbres del idioma en que sea escrita.
Debe prestarse especial atenci\'on al guionado de las palabras, debido a que muchas aplicaciones inform\'aticas
usan el propio del ingl\'es y no el del castellano.

En cuanto a la portada, se debe utilizar la portada oficial de la ETSII. Un ejemplo de la misma puede encontrarse al 
principio (como portada) de este documento.


 
