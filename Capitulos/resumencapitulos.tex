\chapter{Gu\'ia de cap\'itulos de desarrollo}\label{guiacapitulos}

En los capítulos siguientes se hablará sobre el desarrollo del sistema. Los capítulos estarán en orden cronológico explicando primero lo que se ha desarrollado antes y por qué.

En este punto el estudio previo ya se ha realizado, por lo que la mayoría de los capítulos estarán enfocados a desarrollar un sistema inicial y mejorarlo de forma iterativa. 

Los capítulos son los siguientes:

\begin{itemize}
\item \textbf{De computación local a computación en la nube.} Se prueba un ejemplo de segmentación en local y, después de ver las limitaciones de hardware, se pasa a una opción en la nube.
\item \textbf{Sistema inicial.} Se construye un sistema inicial que es capaz de producir un resultado. Para esto se usa una implementación simplificada de la arquitectura U-Net, además de elegir métricas, función de pérdida y otros parámetros.
\item \textbf{Mejora 1: Función de pérdida.} 
\item \textbf{Mejora 2: Data Augmentation.} Se aplican técnicas de data augmentation a los datos de entrada para aumentar el nº de ejemplos de entrenamiento de forma artificial.
\item \textbf{Mejora 3: Arquitectura U-Net completa.} Se cambia la arquitectura del modelo por la U-Net completa.
\item \textbf{Mejora 4: Apex, precisión mixta.} Se investiga sobre la precisión de los tensores y cómo cambiarla para aumentar la velocidad de entrenamiento y reducir el consumo de memoria.
\item \textbf{Mejora 5: DT Watershed.} Se utiliza un enfoque distinto en el que se aplica el algoritmo watershed a la salida.
\end{itemize}